\documentclass{article}

\usepackage{amsmath}

\title{World coordinates from OpenCV camera calibration}
\author{Alex Grant}
\date{16 January 2024}



\begin{document}

\maketitle

\section{Calibration}

The OpenCV \texttt{cameraCalibration} function (and pose estimation, \texttt{solvePnP}) return a rotation vector $r$ and translation vector $t$. These specify the change of coordinates from the coordinate system of the calibration target to that of the camera.

The rotation vector is an encoding of the $3\times 3$ rotation matrix, which can be recovered via
\begin{verbatim}
R = cv2.Rodrigues(r)
\end{verbatim}

Let $p$ be a point in world coordinates, then this point in camera coordinates is given by
\begin{equation}
  \label{eq:camera-coords}
  q = Rp+t
\end{equation}
As a corollary, we can find the position $p_0$ of the camera origin in world coordinates, solving~\eqref{eq:camera-coords} with $q=0$,
\begin{equation*}
  Rp_0 + t = 0
\end{equation*}
threrefore $p_0 = -R^{-1}t$.
Rotation matrices are orthogonal, i.e. $R^t R = I$ and hence $R^{-1} = R^t$. Thus the coordinates of the camera in the basis of the calibration taget (where $(0,0,0)$ is the planar calibration target aligned with the $z$-axis) are
\begin{equation*}
  -R^t t
\end{equation*}

Now consider multiple cameras, with rotation matrices $R_i$ and translation vectors $t_i$, $i=0,1,\dots$ where calibration has been performed for each of them based on images taken of the same calibration target.

We can find the relative position $v$ of camera $i$ with respect to camera $0$ (expressed in the world coordinate basis) as follows:
\begin{equation*}
  v_i = R_0^t t_0 - R_i^t t_i
\end{equation*}

If we want this in the camera 0 basis, we can transform the position of camera $i$ in world coordinates, $-R_i^t t_i$ using~\eqref{eq:camera-coords} to obtain $-R_0 R_i^t t_i +t_0$.


\end{document}